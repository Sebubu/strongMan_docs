\subsection{Nichtfunktionale Anforderungen}
\subsubsection{Funktionalität}
\paragraph{Sicherheit}
Änderung an den Daten der Anwendung dürfen nur authentifizierte Benutzer machen. Nicht authentifizierte Benutzer werden auf die Anmeldeseite weitergeleitet.

\paragraph{Sensitive Daten}
Kritische Daten sind das Passwort für den Benutzer/Adminstrator, die Private Keys und Secret Konfigurationsdaten (z.B. EAP Passwort). Diese müssen verschlüsselt in der Datenbank gespeichert werden.

\paragraph{Passwort Anforderungen}
\begin{itemize}
	\item minimal 8 Zeichen
	\item maximal 60 Zeichen
	\item minimal 1 Sonderzeichen und 1 Zahl
	\item minimal 1 Grossbuchstaben
	\item minimal 1 Kleinbuchstaben
\end{itemize}

\paragraph{Weitere Schutzmassnahmen}
\begin{itemize}
	\item Die Applikation muss vor SQL Injection sicher sein
	\item Es sollen CSFR Tokens verwendet, um vor Cross Site Attacken sicher zu sein.
\end{itemize}

\paragraph{Interoperabilität}
Als externe Schnittstelle ist das VICI Interface aufzuzählen, dabei wir über Unix Sockets kommuniziert. strongSwan bietet dazu schon eine Python Implementation an.
\paragraph{Richtigkeit}
Alle Eingabefehler müssen validiert und auf deren Korrektheit überprüft werden.

\subsubsection{Zuverlässigkeit}

\paragraph{Wiederherstellbarkeit}
Nach einem Systemabsturz oder Stopp, soll die Anwendung ohne Komplikationen wieder gestartet werden können. Der User muss sich neu einloggen.

\paragraph{Fehlertoleranz}
Bei einem Anwendungsfehler sollte nur die Operation des verursachten Benutzers abgebrochen werden. Andere Benutzer sollten vom Fehler nicht betroffen sein.\\
Bsp.: Will man ein Zertifikat löschen, welches noch von einer Verbindungskonfiguration verwendet wird, ist dieses nicht löschbar.

\subsubsection{Benutzbarkeit}

\paragraph{Erlernbarkeit}
Neue Benutzer sollten die grundlegenden Funktionen durch Verwenden der Anwendung und eines Tutorials erlernen können.

\paragraph{Bedienbarkeit}
Die Applikation sollte sowohl auf einem Smartphone, wie auch auf einem Desktopcomputer bedienbar sein, weshalb wir mit einem responsiven Webdesign arbeiten. \\\\
Minimale Auflösung: 	640 × 960 px \\
Maximale Auflösung: 	1920 x 1080 px \\\\
Wir verwenden dabei zwei Skalierungsstufen (Smartphones, Desktopcomputer).

\paragraph{Robustheit}
Jedes Eingabefeld soll durch die Webseite validiert werden. Falsche Eingaben werden angezeigt, damit der Benutzer seine Eingabe korrigieren kann. Für jede falsche Eingabe soll eine kleiner Hilfetext erscheinen.

\paragraph{Effizienz}
Die Anwendung wird für einen zur gleichen Zeit arbeitenden User spezifiziert. Es sollten theoretisch mehrere User auf der Anwendung arbeiten können, es existiert aber nur ein Userlogin und es können im schlimmsten Fall Race Conditions auftreten.
Die Ladezeit für die Anwendung soll für jede Operation unter einer Sekunde bleiben. Bezüglich Geschwindigkeit muss das Projekt nicht spezifisch auf “langsame Devices” wie Router oder Raspberry Pi's achten.

\subsubsection{Änderbarkeit}
Die Applikation soll auch nach Projektabschluss betreubar sein. Daher sollen folgende Anforderungen erfüllt werden: 
\begin{itemize}
	\item Der Coding-Standard PEP8 für Python soll eingehalten werden.
	\item Klassen und nicht-triviale Methoden sollen im Code dokumentiert werden (Doc-Strings)
\end{itemize}

\paragraph{Modifizierbarkeit}
Es soll bei der Entwicklung darauf geachtet werden, zukünftige Erweiterungen nicht durch Design-Entscheide auszuschliessen.


\subsubsection{Übertragbarkeit}
Server: Die Anwendung muss mithilfe eines Django kompatiblen Webservers zum laufen gebracht werden. Zusätzlich zu den Django Anforderungen muss strongSwan mit der VICI-Schnittstelle installiert sein.

Client: Auf dem Client muss ein aktueller Browser mit HTML5 und Javascript Unterstützung installiert sein. Die Applikation wird mit Google Chrome 49 und Firefox 45 getestet.


\paragraph{Internationalization}
Die Applikation ist in der Englischen Sprache verfügbar.

\paragraph{Testing}
Die Hauptfunktionen im Projekt sollten mittels Unit Tests getestet werden. Die Haupt Use Cases müssen mit Integration Tests überprüft werden. Es sind keine Stresstests mit Loadtesting Frameworks wie Gatling vorgesehen.



