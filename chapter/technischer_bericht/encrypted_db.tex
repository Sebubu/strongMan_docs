\subsection{Verschlüsselte Datenbankfelder}
Während des Projekts ist die Anforderung von der Verschlüsslung der Datenbank aufgekommen. Die Datenbank verfügt über sensible Daten wie Private Keys und Passwörter, die nicht einfach offen gelesen werden dürfen.\\


\decision{Datenbankfelder werden mit einem Secret Key verschlüsselt}
Nach einer Besprechung mit unserem Betreuer, haben wir uns auf die Verschlüsslung von einzelnen sensiblen Datenbankfelder mit einem Secret Key entschieden. Der Secret Key muss bei jeder Neuinstallation der Applikation zufällig generiert werden.\\


Das Projekt  Django Fernet-Fields\footnote{\url{https://github.com/orcasgit/django-fernet-fields}} stellt ein angepasstes Datenbankfeld EncryptedField zur Verfügung, das als Django ORM Feld verwendet werden kann. Django kann Klartext in EncryptedField schreiben und wieder Klartext daraus lesen. EncryptedField übernimmt das ganze Verschlüsslungsmanagement. In der Datenbank sind nur kryptische Zeichen zu finden.

\subsubsection{Anpassung des Fernet-Fields}
Das Fernet-Field Projekt benötigt die Python Bibliothek Cryptography\footnote{\url{https://cryptography.io/en/latest/fernet/}}, welche wiederum C++ Code während der Installation kompilieren muss. Um dies zu vermeiden und um die Installation zu vereinfachen, haben wir die EncryptedField Klasse leicht angepasst, damit diese pyaes\footnote{\url{https://github.com/ricmoo/pyaes}} nutzt und wir die Abhängigkeit zur Cryptography verlieren.

Beim ersten Start von strongMan generiert base.py einen neuen Secret Key\footnote{https://gist.github.com/ndarville/3452907} und schreibt diesen in die Datei db\_key.txt.
EncryptedField benutzt diesen Secret Key und verschlüsselt die Daten mit AES256.