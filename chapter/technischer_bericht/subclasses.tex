\subsection{Subklassen}
\label{subklassen}
\subsubsection{Ausgangslage}
Um die diversen Authentiserungsmethoden sowohl in der Datenbank, wie auch in unseren Models widerzuspiegeln, setzen wir auf Vererbung. Wird in Django Vererbung beim Models eingesetzt, wird standardmässig \textbf{Multi-table inheritance} verwendet. Das heisst, es gibt für jede Subklasse eine weiter Tabelle in der Datenbank.

\subsubsection{Beispiel Vererbung}
Beispiel Code in Django um die Vererbung zu demonstrieren.
\medskip
\begin{python}
from django.db import models

class Authentication(models.Model):
    name = models.TextField()
    round = models.IntegerField(default=1)

class EapAuthentication(Authentication):
    eap_id = models.TextField()
\end{python}

\medskip
Dies führt zu folgenden Tabellen in der Datenbank:\\

\begin{table}[H]
	\centering
    \begin{tabular}{|p{3cm}|p{3cm}|p{3cm}|}
    \hline    
    \rowcolor{lightblue}
	id & name & round \\ \hline   
	1 & eap home & 1 \\ \hline
	2 & cert work & 1 \\ \hline
    \end{tabular}
    \caption[Authentication]{Authentication}
\end{table}
\vspace*{1 cm}
\begin{table}[H]
	\centering
    \begin{tabular}{|p{4.5cm}|p{4.5cm}|}
    \hline    
    \rowcolor{lightblue}
	authentication\_ptr\_id & eap\_id \\ \hline   
	1 & eap-home-id  \\ \hline
    \end{tabular}
    \caption[EapAuthentication]{EapAuthentication}
\end{table}
\medskip

\subsubsection{Problematik}
Ein Django Fremdschlüssel referenziert immer auf eine spezifische Klasse. Im Falle einer Klasse mit Subklassen referenziert der Fremdschlüssel somit immer auf die Basisklasse. Die Auflösung des Fremdschlüssels gibt immer die Basisklasse zurück, nicht direkt die Subklasse. Aus diesem Grund musste ein Workaround implementiert werden, der die Basisklasse wenn möglich in die Subklasse umwandelt.

Das gleiche Problem ergibt sich, wenn anhand der ID aus einem Formular auf das spezifische Objekt zugegriffen werden muss.
\\
Dies könnte wie folgt aussehen. CaCertificateAuthentication ist eine Subklasse von Authentication.
\medskip
\begin{python}
    def check_instance(auth):
        print("is instance of Authentication: " + 
            str(isinstance(ca_auth, Authentication)))
        print("is instance of CaCertificateAuthentication: " + 
            str(isinstance(ca_auth, CaCertificateAuthentication)))
\end{python}
Die Funktion \textbf{check\_instance} macht entsprechende Instanzprüfungen.\\

\begin{python}
    ca_auth = CaCertificateAuthentication()
    ca_auth.save()
    check_instance(ca_auth)
\end{python}

\begin{lstlisting}[style=BashInputStyle]
    is instance of Authentication: True
    is instance of CaCertificateAuthentication: True
\end{lstlisting}    
Erstellt ein  CaCertificateAuthentication Objekt und speichert es in der Datenbank ab. Wie an der Ausgabe ersichtlich ist, ist es vom Typ der Subklasse CaCertificateAuthentication.\\

\begin{python}
    loaded_auth = Authentication.objects.get(id=ca_auth.id)
    check_instance(loaded_auth)
\end{python}
\begin{lstlisting}[style=BashInputStyle]
    is instance of Authentication: True
    is instance of CaCertificateAuthentication: False
\end{lstlisting} 
Das Objekt wird nun wieder aus der Datenbank gelesen. Die Ausgabe zeigt, dass es sich nun nur um die Basisklasse und nicht um die Subklasse handelt. Die Umwandlung in die Subklasse ist somit Aufgabe des Programmierers.\\
Das geladene Objekt ist von Typ Authentication. Falls hinter dieser \textbf{id} jedoch noch eine Spezialisierung stecken würde, sieht man dies dem Objekt nicht an.


\subsubsection{Lösungsansatz}
Die Lösung für dieses Problem bietet Django selbst leider nicht an. Jedoch ist eine elegante Möglichkeit die Spezialisierung des Objektes über Reflection zu lösen. Dazu implementieren wir in den Basisklassen die Methode \textbf{subclass()}.\\
Diese ermöglicht es für jede Subklasse der Basisklasse zu testen, ob das aktuelle Objekt eine Spezialisierung ist oder nicht, falls ja, wird ein neue spezialisierte Instanz erstellt und zurück gegeben.
\medskip
\begin{python}
class Authentication(models.Model):
    @classmethod
    def get_types(cls):
        ```
        return: A list of all subclasses
        ```
        subclasses = [subclass() for subclass in cls.__subclasses__()]
        return [subclass.get_typ() for subclass in subclasses]	
	
    def subclass(self):
        ```
        return: The specific subclass 
        ```
        for cls in self.get_types():
            authentication_class = getattr(sys.modules[__name__], cls)
            authentication = authentication_class.objects.filter(id=self.id)
            if authentication:
                return authentication.first()
        return self
\end{python}
\medskip
Mit Hilfe dieser Erweiterung sieht das verbesserte obig Beispiel nun folgendermassen aus.
\medskip
\begin{python}
def update(request, id):
    authentication = Authentication.objects.get(id=id).subclass()
    ...
\end{python}
\medskip
Der Typ des \textbf{authentication} Objektes entspricht nun der gewünschten Subklasse.\\\\


