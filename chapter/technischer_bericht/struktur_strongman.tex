\subsection{Struktur strongMan}
Die strongMan Projektstruktur ist im Anschluss aufgeführt, sie beinhaltet nur die strukturell wichtigsten Punkte.\\
\begin{figure}[H]
\dirtree{%
.1 strongMan.
.2 apps.
.3 connections.
.3 certificates.
.3 vici.
.3 encryption.
.2 settings.
.2 fixtures.
.2 tests.
}
\end{figure}

\medskip

\subsubsection{apps}
\par
\begingroup
\leftskip=0.5cm 
\noindent
\paragraph{vici} stellt den anderen Apps den Einstiegspunkt zur Vici-Schnittstelle zur Verfügung. 

\paragraph{encryption} ist für die Verschlüsselung der sensitiven Daten verantwortlich.

\paragraph{certificates} verwaltet die Zertifikate, regelt somit das Erfassen, Löschen, Entschlüsseln und Persistieren. Weiter bestehen Abhängigkeiten zu der \textbf{vici} App, welche Zertifikate die von der strongSwan Applikation
verwaltet werden auch in strongMan einbindet und zur \textbf{encryption}, die die privaten Schlüssel verschlüsselt in der Datenbank ablegt.

\paragraph{connections} verwaltet die Konfiguration der Verbindungen, sowie das Aufbereiten dieser Daten in VICI/strongSwan konforme ordered Dictionaries. Es gibt also Abhängikeiten zu allen anderen Apps.

\par
\endgroup

\subsubsection{settings}
Beinhaltet die Django spezifischen Einstellung, wie zum Beispiel die verwendeten Apps, Environement Variabeln, Modies etc. Besteht aus verschieden Settingsfiles, die je nach Verwendung in Einsatz kommen. Namentlich:
\begin{itemize}
    \item base.py, dient als Basis für alle anderen
    \item local.py, für die Entwicklung
    \item deployment.py, gedacht für die produktiven Verwendung 
\end{itemize}

\subsubsection{fixtures}
Wird genutzt um Initialdaten in Form einer Json-Datei der Datenbank zur Verfügung zu stellen. In unserem Fall wird der vorkonfigurierte Benutzer und dessen Passwort gesetzt.

\subsubsection{tests}
Ordner der die Unit-Tests, sowie die Integrations-Tests beinhaltet. Diese können durch Strukturierung seperat ausgeführt werden.
\newpage