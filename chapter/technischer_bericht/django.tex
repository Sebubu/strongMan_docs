\subsection{Einführung Django}
Django \cite{django} ist ein Python Framework zur Webentwicklung. Projekte werden zur Strukturierung in Apps eingeteilt, die hierarchisch aufeinander aufbauen. Apps sollten so gestaltet werden, dass man sie wiederverwenden kann.
\paragraph{MVC Konzept}
Django baut auf dem MVC Konzept auf, unterscheidet sich jedoch in der Namensgebung vom klassischen Aufbau. 
\medskip
\begin{table}[H]
\centering
    \begin{tabular}{|l|l|l|}
    \hline    
    \rowcolor{lightblue}
	klassisch & Django & Anmerkung \\ \hline
	Model & Model & Models sind gleich wie allgemein üblich \\ \hline
	View & Templates & Views sind in Django Templates, diese sind nicht mehr als Vorlagen \\ \hline
	Controller & Views & Der Controller wird zur View und dient als Schnittstelle für den Client \\ \hline
    \end{tabular}
    \caption[MVC Konzept Django]{MVC Konzept Django}
\end{table}

\paragraph{Templates}
In Django ist ein Template im Grund eine einfach Textdatei, welches \textbf{variables} und \textbf{tags} enthält. Die \textbf{\{\{ variables \}\})} werden durch die entsprechenden Werte ersetzt und die \textbf{\{\% tags \%\}} beinhalten die Logik der Templates.

\subsubsection{Beispiel Template}
\begin{python}



	
	
		Daemon: {{ daemon }} <br>
        <h2>Installed plugins</h2>
       	
        	{{ plugin }}
        
	
    	<h3>Can not display information without the vici interface.</h3>
    

\end{python}

\paragraph{Views} Die Views verarbeiten die Requests, sammeln Daten aus der Datenbank, bestücken die Templates und retournieren diese.

\subsubsection{Beispiel views.py}
\begin{python}
from django.shortcuts import render
from django.contrib.auth.decorators import login_required

@login_required
def overview(request):
        connections = Connection.objects.all()
        return render(self.request, 'overview.html', {'connections': connections})
\end{python}

\paragraph{Models} Django Models definieren die Daten. Sie beinhalten die entsprechenden Felder und Relation, regeln die Zugriffe auf die Datenbank mithilfe eines OR-Mappers und beinhalten das Verhalten der Daten.
\begin{itemize}
	\item Ein Model steht für eine Tabelle in der Datenbank
	\item Die Attribute der Models widerspiegeln die Spalten
	\item Und die Methoden repräsentieren das Verhalten
\end{itemize}

\subsubsection{Beispiel models.py}
\begin{python}
from collections import OrderedDict
from django.db import models

class Secret(models.Model):
    type = models.TextField()
    data = fields.EncryptedCharField(max_length=50)
    authentication = models.ForeignKey(Authentication)

    def dict(self):
        eap_id = self.authentication.subclass().eap_id
        return OrderedDict(type=self.type, data=self.data, id=eap_id)
\end{python}


\paragraph{Routing} Um Adressen auf Funktionen und schlussendlich Seiten zu mappen, verwendet Django URL-Konfigurationsdateien (urls.py). Diese enthalten reguläre Ausdrücke, welchen Funktionen aus Module zugeordnet 
werden.

\subsubsection{Beispiel urls.py}
\begin{python}
from django.conf.urls import url

app_name = 'connections'
urlpatterns = [
    url(r'^$', views.overview, name='index'),
    url(r'^add/$', views.create, name='choose'),
    url(r'^(?P<id>\d+)/$', views.update, name='update')
]
\end{python}
\newpage