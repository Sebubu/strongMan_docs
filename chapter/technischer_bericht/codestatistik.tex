\section{Codestatistik}
\subsubsection{Test Coverage}
Test Coverage wurde mithilfe von \textbf{Pycharm} gemessen. \\

\begin{table}[H]
\centering
    \begin{tabular}{|p{6cm}|p{6cm}|}
    \hline    
    \rowcolor{lightblue}
	App & Coverage \\ \hline
	connections & 88\% \\ \hline    
	certificates & 88\% \\ \hline   
	vici & 88\% \\ \hline  
	encryption & 94\% \\ \hline  
	\rowcolor{lightblue}
	Durchschnitt &   90\% \\ \hline
    \end{tabular}
    \caption[Test Coverage]{Test Coverage}
\end{table}

\subsubsection{Codezeilen}
Die Codezeilen wurden mit Hilfe von \textbf{CLOC} \cite{CLOC} gezählt. Die Anzahl Zeilen beinhalten nur selbst geschrieben Code. Externe Bibliotheken wurden ausgeschlossen. \\

\begin{table}[H]
\centering
    \begin{tabular}{|p{3cm} |p{3cm} |p{3cm} |}
    \hline    
    \rowcolor{lightblue}
	Sprache & Dateien & Zeilen  \\ \hline   
	Python & 70 & 5508 \\ \hline
	HTML & 41 & 1857 \\ \hline
	JavaScript & 4 & 397 \\ \hline
    \end{tabular}
    \caption[Codezeilen]{Codezeilen}
\end{table}

