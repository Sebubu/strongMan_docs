\section{Codestatistik}
\subsection{Test Coverage}
Test Coverage wurde mit \textbf{Pycharm} durchgeführt. \\

\begin{table}[H]
\centering
    \begin{tabular}{|p{6cm}|p{6cm}|}
    \hline    
    \rowcolor{lightblue}
	App & Coverage [\%] \\ \hline
	connections & 88 \\ \hline    
	certificates & 88 \\ \hline   
	vici & 88 \\ \hline  
	encryption & 94 \\ \hline  
	\rowcolor{lightblue}
	Durchschnitt &   90 \\ \hline
    \end{tabular}
    \caption[Test Coverage]{Test Coverage}
\end{table}

\subsection{Codezeilen}
Die Codezeilen wurden mit Hilfe von \textbf{CLOC} \cite{CLOC} ausgezählt. \\

\begin{table}[H]
\centering
    \begin{tabular}{|p{3cm} |p{3cm} |p{3cm} |}
    \hline    
    \rowcolor{lightblue}
	Sprache & Dateien & Zeilen  \\ \hline   
	Python & 43 & 2045 \\ \hline
    \end{tabular}
    \caption[Codezeilen]{Codezeilen}
\end{table}

