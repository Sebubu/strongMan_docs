\newpage
\section{Resultate}
\subsection{Zielerreichung}
-
\subsubsection{Ziel der Aufgabenstellung}
\begin{table}[H]
    \begin{tabular}{|p{6cm}|p{8cm}|}
    \hline    
    \rowcolor{lightblue}
	Ziel & Resultat \\ \hline
	Evaluation eines effizienten Algorithmus zur Erkennung von Fussgängerstreifen auf Orthofotos. & Diverse Algorithmen wurden evaluiert, mit dem Deep Learning Ansatz wurde ein klarer Favorit ermittelt. Mehr dazu unter dem Abschnitt~\ref{sec:suchalgorithmus} auf der Seite~\pageref{sec:suchalgorithmus}.\\ \hline
	Automatische Verarbeitung von Orthofotos. & Es wird automatisch auf Bilder von Bing Maps zugegriffen.\\ \hline
	Extraktion der Koordinaten von Fussgängerstreifen aus Orthofotos (Kanton Zürich, optional Europa oder mehr). & Zürich konnte von der Applikation verarbeitet werden, weiter wurde die Suche auf die Ostschweiz ausgebaut. \\ \hline
	Evaluation des Crowdsourcing-System zur Daten-Validierung und Übertragung in OSM.& Bei der Evaluation des Crowdsourcing-Systems setzte sich MapRoulette durch die Bekanntschaft bei der Community durch. Mehr dazu unter dem Abschnitt~\ref{sec:crowdsourcing} auf der Seite~\pageref{sec:crowdsourcing}. \\ \hline
	Erstellung einer Challenge für das Crowdsourcing-System anhand der gesammelten Daten. & Eine Challenge wurde für MapRoulette generiert und publiziert.\\ \hline
    \end{tabular}
    \caption[Resultate]{Resultate}
\end{table}
Die in der Aufgabenstellung formulierten Ziele konnten alle in einem angemessenem Rahmen erreicht werden. 

\newpage
\subsection{Persönlicher Bericht}
Mit dem Resultat des Projektes sind wir äusserst zufrieden. Es wurde viel neues dazugelernt, sowohl in technischen Bereichen, wie auch in der Teamkommunikation und dem Projektmanagement.
\subsubsection{Neu erlernte Technologien}
\begin{itemize}
	\item Python
	\item \Gls{Docker}
	\item Latex
\end{itemize}
\newpage

\subsection{Dank}
-