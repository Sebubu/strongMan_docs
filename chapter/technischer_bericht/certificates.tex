\subsection{Zertifikate}
Um IPsec Verbindungen mit asymmetrischen Schlüsseln zu erstellen, müssen Schlüssel verwaltet werden. strongMan persistiert geuploadeten Schlüsselcontainer in der Datenbank.
Dabei werden folgende Schlüsselcontainer Formate unterstützt: \\

\begin{tabular}{ | p{0.2\textwidth} | p{0.7\textwidth} | }
\hline
    Schlüsselcontainer: & X509, PKCS1, PKCS8, PKCS12 \\ \hline
    Encoding: & PEM, DER \\ \hline
    Algorithmen: & RSA, ECDSA \\ \hline
\end{tabular}
\\\\
Die Applikation hat ein Uploadfeld um Schlüsselcontainer hinzuzufügen. Dabei müssen die vier verschiedenen Containertypen unterschieden werden, um die Daten richtig auszulesen. Die Klasse ContainerDetector probiert die Datei mit allen vier ASN1 Strukturen zu parsen und gibt bei Erfolg den entsprechenden Typ zurück.

\subsubsection{Wichtige Implementationsmerkmale}
\paragraph{Private Keys} Wie im Domain Model zu erkennen ist, hängt der Private Key immer an einem UserCertificate. Das Zertifikat muss somit immer vor dem Private Key geuploadet werden. Dies verhindert Private Key Leichen in der Datenbank und vereinfacht das Schlüsselhandling.

\paragraph{Doppelte Einträge} Der CNAME ist normalerweise ein einfacher Eintrag wie www.raiffeisen.ch. In Ausnahmefällen kann der CNAME, oder auch andere Felder, eine Liste wie ['raiffeisen.ch', 'www.raiffeisen.ch'] sein. Tritt dieser Fall ein, wird die Liste in einen String umgewandelt und so dargestellt.

\paragraph{Doppelte Zertifikate/Private Keys}
Werden bestehende Zertifikate oder Private Keys nochmals geuploadet, erkennt die Zertifikatsverwaltung dies und verweigert den Upload. Das Matching wird anhand des Public Keys vorgenommen. Bei Zertifikaten wird der Public Key und die Version gematcht.

\paragraph{Zertifikate in Gebrauch}
Ist ein Zertifikat in einer Verbindungskonfiguration ausgewählt, so kann das Zertifikat nicht gelöscht werden. Das Zertifikat muss zuerst deselektiert werden, um es aus der Zertifikatsverwaltung zu löschen.

\newpage



