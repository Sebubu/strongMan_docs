\section{Evaluation Usermanagement}
Da mit Hilfe von stongSwan VPN Tunnels zu schützenswerten Daten ermöglicht werden, sollt auch die Applikation strongMan mit einer Authentifizierung versehen werden.
\subsubsection{Anforderungen}
\begin{itemize}
	\item Benutzer hat ohne sich einzuloggen keinen Zugriff auf die Applikation
	\item Benutzer hat die Möglichkeit sein Passwort zu ändern
\end{itemize}

\subsubsection{Varianten}
\begin{itemize}
	\item Ein vorkonfigurierter Benutzer
	\item Mehrere Benutzer
	\item Benutzer des Systems verwenden
\end{itemize}
\subsubsection{Auswertung}
\begin{table}[H]
\centering
    \begin{tabular}{|l|l|l|l|}
    \hline
    \rowcolor{lightblue}
    Variante & Implementierung & Framework Unterstützung & Komplexität   \\ \hline
	Ein vorkonfigurierter Benutzer	&	leicht	& gegeben	&	niedrig	\\ \hline
		Mehrere Benutzer	&	mittel	& gegeben	&	mittel	\\ \hline
		Benutzer des Systems verwenden	&	schwer	& nicht gegeben	&	hoch	\\ \hline	
	\end{tabular}
    \caption[Evaluation Usermanagement]{Evaluation Usermanagement}
\end{table}

\decision{Evaluation Usermanagement}
Durch das Vergleichen der verschieden Varianten und in Absprache mit den Betreuern hat sich 'Ein vorkonfigurierter Benutzer' als klarer Favorit herausgestellt. Dabei wir ein Standard Benutzer während er Installation der Applikation erstellt und dessen Passwort kann im Anschluss geändert werden.

\section{Evaluation Zertifikatsbibliothek}
Um VPN-Verbindungen mit Zertifikaten abzusichern, muss das strongSwan Management Interface mit Zertifikaten umgehen können. Dabei müssen verschiedene Formate unterstützt, ausgelesen und entsprechend Ihrem Inhalt an die Vici-Schnittstelle geliefert werden.
\subsubsection{ASN.1 Schemas}
In der Welt der asymmetrischen Verschlüsslung werden alle nötigen Daten mithilfe einer ASN.1 Struktur abgelegt. Der ASN.1 Standard stellt Datentypen und Datenstrukturen zur Verfügung, um Informationen mithilfe eines Schemas abzulegen. Über die Zeit sind verschiedene RFC’s entstanden, die ASN.1 Schemas definieren. Unten aufgelistet sind die Standards, welche wir in unserem Projekt berücksichtigen. \medskip

\begin{table}[H]
\centering
    \begin{tabular}{|l|p{12cm}|l|}
    \hline
    \rowcolor{lightblue}
    Standard & Inhalt & Forderung   \\ \hline
	PKCS\#1	&	Private Schlüssel der verschiedene Algorithmen RSA, DSA, ECDSA. Können mit einem Passwort verschlüsselt sein.	& Muss \\ \hline	
		PKCS\#8	&	Gleich wie PKCS\#1, nur hier in einer besser verständlichen Form.	& Muss \\ \hline	
		X.509	&	Zertifikatsstandard mit allen Informationen zum Eigner. Unverschlüsselt.	& Muss \\ \hline
		PKCS\#12	&	Container mit Zertifikaten und Private Keys. Kann verschlüsselt sein.	& Kann \\ \hline		
	\end{tabular}
    \caption[ASN.1 Schemas]{ASN.1 Schemas}
\end{table}

\subsubsection{Encoding}
Die ASN.1 Strukturen werden nicht direkt in die Dateien abgespeichert, sondern zuerst in einem speziellen Format codiert. Dabei sind zwei verschiedene Standards üblich. \medskip

\begin{table}[H]
\centering
    \begin{tabular}{|l|p{12cm}|l|}
    \hline
    \rowcolor{lightblue}
    Standard & Inhalt & Forderung   \\ \hline
	DER	&	Binär kodiert und somit nicht in einem Editor lesbar. Dateiendung .der.	& Muss \\ \hline	
		PEM	&	
Base64 encodet in ASCII Zeichen. Dateiendung .pem. Openssl nutzt PEM als Standardencoding.	& Muss \\ \hline		
	\end{tabular}
    \caption[Encoding]{Encoding}
\end{table}

\subsubsection{Verschlüsslung}
Die Dateien mit Private Keys (PKCS\#1, PKCS\#8 und PKCS\#12) können verschlüsselt sein. Dabei kommen verschiedene symmetrische Verschlüsslungen zum Einsatz, die mit einem Userpasswort entschlüsselt werden.


\subsubsection{Anforderungen}
In dieser Evaluation soll eine Python Crypto Bibliothek gefunden werden, mit der man grundlegende Informationen wie CNAME und ASN.1 Schema auslesen kann. Mithilfe der Anforderungstabelle unten soll eine passende Bibliothek gefunden werden.

\paragraph{Muss Anforderungen}
\begin{itemize}
	\item Unterscheidung Private Key / Zertifikat
	\item Erkennung von CA Zertifikaten
	\item Unterstützt alle obigen ASN.1 Schemas
	\item Unterstützt PEM und DER Encoding
	\item Aus X.509 können Infos wie der CNAME ausgelesen werden	
\end{itemize}

\paragraph{Kann Anforderungen}
\begin{itemize}
	\item Private Keys können entschlüsselt werden	
	\item Bibliothek ist pure Python (Keine Kompilation bei Installation)
	\item PKCS\#12 wird unterstützt
	\item Matching zwischen Public und Private Key ist möglich
	\item Einfaches API
\end{itemize}

\subsubsection{Kandidaten}
\subsubsection{Pyasn1}
\begin{table}[H]
\centering
    \begin{tabular}{|p{12cm}|l|l|}
    \hline
    \rowcolor{lightblue}
    Anforderungen & Muss / Kann & Erfüllt   \\ \hline
	Unterscheidung Private Key / Zertifikat	&	Muss & x \\ \hline	
	Erkennung von CA Zertifikaten	&	Muss	& x \\ \hline	
	Unterstützt alle obigen ASN.1 Schemas	&	Muss	& x \\ \hline		
	Unterstützt PEM und DER Encoding	&	Muss	& x \\ \hline	
	Aus X.509 können Infos wie der CNAME ausgelesen werden &	Muss	& x \\ \hline	
	Private Keys können entschlüsselt werden &	Kann &  \\ \hline
	Bibliothek ist pure Python (Keine Kompilation bei Installation) &	Kann	&  x \\ \hline
	PKCS\#12 wird unterstützt &	Kann &  \\ \hline
	Matching zwischen Public und Private Key ist möglich &	Kann &  \\ \hline
	Einfaches API &	Kann &  \\ \hline
	\end{tabular}
    \caption[Pyasn1]{Pyasn1}
\end{table}

Pyasn1 stellt einzelnen ASN.1 Schemas für die dekodierung von ASN.1 Dateien zur Verfügung. Einzelne Standards wie X.509 sind schon implementiert, jedoch ist man selbst verantwortlich, das richtige Schema zu der entsprechenden Datei zu finden. Das API ist sehr kompliziert und der Programmieraufwand für die Unterstützung alles nötigen Formate ist gross.


\subsubsection{Cryptography}
\begin{table}[H]
\centering
    \begin{tabular}{|p{12cm}|l|l|}
    \hline
    \rowcolor{lightblue}
    Anforderungen & Muss / Kann & Erfüllt   \\ \hline
	Unterscheidung Private Key / Zertifikat	&	Muss &  \\ \hline	
	Erkennung von CA Zertifikaten	&	Muss	& x \\ \hline	
	Unterstützt alle obigen ASN.1 Schemas	&	Muss	&  \\ \hline		
	Unterstützt PEM und DER Encoding	&	Muss	&  x\\ \hline	
	Aus X.509 können Infos wie der CNAME ausgelesen werden &	Muss	& x \\ \hline	
	Private Keys können entschlüsselt werden &	Kann &  \\ \hline
	Bibliothek ist pure Python (Keine Kompilation bei Installation) &	Kann	&   \\ \hline
	PKCS\#12 wird unterstützt &	Kann &  \\ \hline
	Matching zwischen Public und Private Key ist möglich &	Kann &  \\ \hline
	Einfaches API &	Kann & x \\ \hline
	\end{tabular}
    \caption[Cryptography]{Cryptography}
\end{table}

Cryptography ist ein Python Bibliothek mit dem Fokus auf komfortabler Cryptography in Python. Es hat eine tolle Unterstützung für X.509 Zertifikate. Man kann Private Keys generieren, ein CSR erstellen und sogar CSR signieren. Im Hintergrund verwendet es OpenSSL. Nichts desto trotz bietet es keinen Support ausserhalb vom X.509 und ist somit unbrauchbar für unsere Zwecke.


\subsubsection{Pycrypto}
\begin{table}[H]
\centering
    \begin{tabular}{|p{12cm}|l|l|}
    \hline
    \rowcolor{lightblue}
    Anforderungen & Muss / Kann & Erfüllt   \\ \hline
	Unterscheidung Private Key / Zertifikat	&	Muss & x \\ \hline	
	Erkennung von CA Zertifikaten	&	Muss	& x \\ \hline	
	Unterstützt alle obigen ASN.1 Schemas	&	Muss	& x \\ \hline		
	Unterstützt PEM und DER Encoding	&	Muss	&  x \\ \hline	
	Aus X.509 können Infos wie der CNAME ausgelesen werden &	Muss	& x \\ \hline	
	Private Keys können entschlüsselt werden &	Kann &  \\ \hline
	Bibliothek ist pure Python (Keine Kompilation bei Installation) &	Kann	&   \\ \hline
	PKCS\#12 wird unterstützt &	Kann &  \\ \hline
	Matching zwischen Public und Private Key ist möglich &	Kann &  x \\ \hline
	Einfaches API &	Kann &  \\ \hline
	\end{tabular}
    \caption[Pycrypto]{Pycrypto}
\end{table}

Pycrypto unterstützt alle geforderten Formate inkl. Verschlüsslung. Es basiert auf der OpenSSL Bibliothek, welche davor zuerst installiert werden muss. Die Anwendung von Pycrypto erfordert jedoch einiges an Code, um jedes Format durchzutesten und um die entsprechenden Formate zu erkennen.


\subsubsection{Oscrypto}
\begin{table}[H]
\centering
    \begin{tabular}{|p{12cm}|l|l|}
    \hline
    \rowcolor{lightblue}
    Anforderungen & Muss / Kann & Erfüllt   \\ \hline
	Unterscheidung Private Key / Zertifikat	&	Muss & x \\ \hline	
	Erkennung von CA Zertifikaten	&	Muss	& x \\ \hline	
	Unterstützt alle obigen ASN.1 Schemas	&	Muss	& x \\ \hline		
	Unterstützt PEM und DER Encoding	&	Muss	&  x \\ \hline	
	Aus X.509 können Infos wie der CNAME ausgelesen werden &	Muss	& x \\ \hline	
	Private Keys können entschlüsselt werden &	Kann &  x \\ \hline
	Bibliothek ist pure Python (Keine Kompilation bei Installation) &	Kann	&  ~ \\ \hline
	PKCS\#12 wird unterstützt &	Kann & x \\ \hline
	Matching zwischen Public und Private Key ist möglich &	Kann & ~  \\ \hline
	Einfaches API &	Kann & x \\ \hline
	\end{tabular}
    \caption[Pycrypto]{Pycrypto}
\end{table}

Oscrypto ist eine neu entstandene Crypto Bibliothek für Python. Es unterstützt alle nötigen Formate inkl. Verschlüsslung. Die meisten Funktionen können ohne die OpenSSL Bibliothek im Hintergrund verwendet werden. Mit wenigen Befehlen kann oscrypto alle nötigen Formate lesen. Es müssen nicht alle durch probiert werden.\\
\medskip

\decision{Evaluation Zertifikatsbibliothek}
Der Entscheid fiehl klar auf oscrypto. Oscrypto ist eine pure Python Bibliothek, welche verschiedene Cryptofunktionen anbietet. Für das Parsen und Lesen von Keycontainer wrappt oscrypto die asn1crypto Bibliothek (auch pure Python). Für andere Cryptofunktionen wrappt oscrypto die standartmässig installierte OpenSSL Library, welche wir aber in dieser Arbeit nicht brauchen.
Mithilfe dieser Library ist nun ein komfortables Arbeiten mit allen oben aufgeführen Keycontainern möglich, ohne je einen nicht-Python Kompiler anwerfen zu müssen. 