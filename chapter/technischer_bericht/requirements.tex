\section{Abhängigkeiten}

\subsection{Python}
Ein Ziel dieses Projektes war es, möglichst keine kompilierende Abhängigkeiten zu verwenden. Die Bibliotheken, die mit Pip installiert werden, dürfen also keinen komplierenden Code wie C++ enthalten. In anderen Worten sollte das Projekt Pure-Python sein. 

\paragraph{django} \cite{django} Bibliothek zur Verwendung des Django Webframeworks.
\paragraph{oscrypto} \cite{oscrypto} Zertifikatsparser.
\paragraph{asn1crypto} Zertifikatsparser, Abhängigkeit von oscrypto. 
\paragraph{pyaes} \cite{pyaes} Python AES Implementation. Gebraucht für die Datenbankverschlüsslung.
\paragraph{django-tables2} \cite{django-tables} Plugin zur Anzeige von HTML Tabellen. Unterstützt Pagination, Sortierung.
\paragraph{vici} \cite{vici} Vici Plugin für strongSwan.
\paragraph{gunicorn} \cite{gunicorn} Webserver für die Produktionsumgebung.
\paragraph{dj\-static} \cite{dj-static} Plugin zur Auslieferung von den statischen Dateien (css, js) in der Produktionsumgebung.


\subsection{Javascript/CSS}
Das Userinterface verwendet einige Javascript und CSS Bibliotheken. Alle Bibliotheken wurden direkt in das Projekt importiert und werden nicht aus dem Internet nachgeladen. Dies dient der Verwendung der Applikation in einer Umgebung mit keiner/beschränkter Internetverbindung.

\paragraph{bootstrap} \cite{bootstrap} CSS Design Bibliothek.
\paragraph{jquery} \cite{jquery} Helfer zur HTML Dom Manipulation.
\paragraph{bootstrap-fileinput} \cite{bootstrap-fileinput} HTML5 Datei Upload Element.
\paragraph{bootstrap-select} \cite{bootstrap-select} Auswahlfeld, gebraucht um Zertifikate in der Connection Config auszuwählen.
\paragraph{bootstrap-toggle} \cite{bootstrap-toggle} Toggle Button, der die Connection startet/stoppt.
\paragraph{callout} \cite{callout} CSS Callout Bibliothek.
\paragraph{font-awesome} \cite{fontawesome} Icon Bibliothek.