\section{Entwicklungsumgebung und Infrastruktur}
\subsubsection{IDE (Integrated Development Environment)}
\decision{PyCharm}	
Beiden Projektmitgliedern ist JetBrains Intellij bekannt und \Gls{PyCharm} ist im Umgang nahe zu identisch.
Für Studenten sind die Entwicklungsumgebungen kostenlos verfügbar.
\subsubsection{SCM (Source Control Management)}
\decision{GitHub}
Der Umgang mit \Gls{Git} ist beiden Projektmitglieder bestens bekannt.
\Gls{GitHub} ist ohne Unkosten von überall verfügbar.
Das Geometalab der HSR publiziert über diesen Weg diverse Projekte.

\subsubsection{CI (Continuous Integration)}
\decision{Travis CI}
Mit \Gls{Travis CI} wurde ein Anbieter gefunden, der die eher komplexeren Anforderungen an das automatisierte Testing des strongMan Projektes erfüllt, sich nathlos in Github integrieren lässt, kostenfrei ist, sowie den Projektmitgliedern schon bekannt ist.

\subsubsection{Projektmanagement Tool}
\decision{Jira}
\Gls{Jira} ist den Projektmitgliedern schon aus der Studienarbeit bekannt und hat sich sehr bewährt.
Das Dashboard ist übersichtlich gestaltet. Es ermöglicht eine Übersicht über die aktuellen Tasks auf einen Blick.
Alle Mitglieder haben jederzeit Zugriff auf die Plattform, was die Transparenz erhöht.
Weiter bietet Jira diverse Reports um Auswertungen über das Projekt zu fahren.
