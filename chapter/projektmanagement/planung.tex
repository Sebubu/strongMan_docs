\section{Planung}
Das Projekt strongMan wird in einer Mischung aus RUP und Agile durchgeführt. Der Projektzeitraum wird zuerst in die RUP Phasen Inception, Elaboration, Construction und Transition aufgeteilt, wobei wir in der Construction agile vorgegangen wird.

\subsection{Phasen}
\begin{enumerate}
  \item Inception
  \begin{enumerate}
    \item Einarbeitung in das Projekt
  \end{enumerate}
  \item Elaboration1
  \begin{enumerate}
    \item Evaluation und Einarbeitung der Techniken (Django, Crypto-Library, Vici-Schnittstelle)
    \item Erstellen der Requirement-Analyse
  \end{enumerate}
  \item Construction1
  \begin{enumerate}
    \item VPN-Tunnel CRUD
    \item VPN-Tunnel starten/stoppen
    \item Login 
    \item Zertifikate CRUD
  \end{enumerate}
  \item Construction2
  \begin{enumerate}
    \item Bestehende Verbindung terminieren
    \item User Password änedern 
    \item Verbindungselemente zwischen VPN-Tunnel und Zertifikate
    \item Verschlüsslung Private Keys und User Password für Tunnels
  \end{enumerate}
    \item Construction3
  \begin{enumerate}
  	\item Erstellen einer Informationsseite
    \item Optional: Administrationsmodus um Serverkonfigurationen vor zu nehmen
  \end{enumerate}
      \item Construction4
  \begin{enumerate}
    \item Applikation finalisieren
    \item Deployment
  \end{enumerate}
  \item Transition
  \begin{enumerate}
    \item Dokumentationsabschluss
  \end{enumerate}
\end{enumerate}
\newpage

\subsection{Meilensteine}
\begin{enumerate}
	\item MS1 Lauffähiges Model
	\item MS2 Techniken evaluiert (Django, Crypto-Library, Vici-Schnittstelle)
	\item MS3 VPN-Tunnel erfass- und änderbar
	\item MS4 VPN-Tunnel initialisier- und terminerbar
\end{enumerate}

\subsection{Zeitplanung}

\begin{table}[H]
	\centering
    \begin{tabular}{|p{6cm}|p{6cm}|}
    \hline    
    \rowcolor{lightblue}
	Phase & Arbeitsaufwand \\ \hline   
	Inception & 1 Woche \\ \hline
	Elaboration & 2 Wochen \\ \hline
	Construction1 & 3 Wochen \\ \hline
	Construction2 & 3 Wochen \\ \hline
	Construction3 & 3 Wochen \\ \hline
	Construction4 & 3 Wochen \\ \hline
	Transition & 1 Woche \\ \hline
	\rowcolor{lightblue}
	Total & 14 Wochen \\ \hline
    \end{tabular}
    \caption[Zeitplanung]{Zeitplanung}
\end{table}
\medskip
Um den Aufwand über die ganze Projektdauer ausgeglichen zu verteilen rechneten wir mit:
\begin{itemize}
    \item Aufwand pro Woche: 52 Stunden
    \item Aufwand geplant Total: 52 Stunden * 14 Wochen = 728 Stunden
    \item Aufwand pro Student: 728 Stunden / 2 = 364 Stunden
\end{itemize}

Das Frühlingssemester 2016 an der HSR hat offiziell 15 Wochen  (inklusive Ostern, Auffahrt und Pfingsten). Bis zur Abgabe der Bachelorarbeit, dem 18.06.2016, sind zusätzlich zwei weitere Wochen gegeben. Damit am Ende es Projektes der Aufwand nicht exponentiell steigt, planten wir unsere Stunden über das offizielle Frühlingssemester und halten uns die restliche Zeit als reserve frei.


