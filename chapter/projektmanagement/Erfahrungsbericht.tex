\section{Erfahrungsbericht}
\subsubsection{Samuel Kurath}
Das Projekt strongMan war einen ausgezeichnete Gelegenheit die während dem Studium erlangten Software Engineering Kenntnisse umzusetzen. Mit der Verwendung einer automatisierten Testumgebung und natürlich den implementierten Unit- wie auch Integrationtests konnten auch grössere Änderungen am Code vorgenommen werden ohne später auf unerwünschte Fehler zu stossen.\\
Weiter hatten wir die Gelegenheit unser Wissen in diversen Technologien auszubauen, dazu zähle ich Python, Schlüsselcontainer und Docker. Mit dem Einsatz von Django einem Python basiertem Webframework und der Docker Erweiterung docker-compose konnte mein persönlicher Toolstack bereichert werden.\\
Die Arbeit am strongMan und die Zusammenarbeit im Team, sowie die Betreuung durch den Experten war immer sehr harmonisch und produktive, was sich aus meiner Sicht auf das positive gelingen des Projektes ausgewirkt hat.


\subsubsection{Severin Bühler}
Während dieses Projekts kamen sehr viel neue oder nur in der Theorie bekannte Technologien wie Django, VirtualEnv, asymmetrische Schlüsselcontainer, SystemD, Webserver auf mich zu. Die Einarbeit in diese war sehr spannend und ich haben einiges gelernt. Auch die vertiefte Arbeit mit Python selbst hat mich noch einige weitere sprachspezifische Eigenheiten lernen lassen.

Nicht desto trotz hatte die Verwendung von Django auch seine negativen Seiten. Das Django ORM hat einen schwerwiegenden Nachteil bezüglich Vererbung (siehe Seite \pageref{subklassen}) und wir hatten einige Mühe diesen zu umgehen. Auch das Django Templating hat seine mühsamen Seiten.

Die Zusammenarbeit mit Herrn Steffen und auch Herr Brunner war zu jedem Zeitpunkt zielorientiert und positiv. Bei Fragen konnten wir immer auf die Unterstützung von Ihnen zählen und wir standen nie unter überzogenem Leistungsdruck. 