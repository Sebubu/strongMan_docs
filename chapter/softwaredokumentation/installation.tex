\section{setup.py}
Die Datei 'setup.py' im Hauptverzeichnis des Projekts stellt das Installationsskript für strongMan dar. Zusätzlich enthält es noch einen migrate Befehl für das automatische Migrieren der Datenbank.

\begin{lstlisting}[style=BashInputStyle]
	 ./setup.py <command> [options]
\end{lstlisting}

\begin{itemize}
	\item -v | --verbose
	\begin{itemize}
	    \item Setzt die Ausgabe von setup.py auf verbose.
	\end{itemize}
\end{itemize}

\subsubsection{install}
\begin{lstlisting}[style=BashInputStyle]
	 ./setup.py install [-p %python-interpreter%]
\end{lstlisting}
Macht strongMan bereit zur Ausführung. Richtet ein Python Virtualenv ein, installiert alle Requirements, migriert die Datenbank und kopiert alle statischen Dateien in einen Ordner.
\begin{itemize}
	\item -p | --python %python-interpreter%
	\begin{itemize}
	    \item Wähle einen spezifischen Python Interpreter für strongMan
	    \item Python3 wird als Standartwert genutzt, wenn nichts angegeben.
	\end{itemize}
\end{itemize}

\subsubsection{uninstall}
\begin{lstlisting}[style=BashInputStyle]
	 ./setup.py uninstall
\end{lstlisting}
Versetzt die strongMan Applikation in den gleichen Zustand wie vor der Installation. Entfernt das Virtualenv und löscht die statischen Dateien.

\subsubsection{add-service}
\begin{lstlisting}[style=BashInputStyle]
	 sudo ./setup.py add-service
\end{lstlisting}
Fügt einen Systemd service für strongMan hinzu. Der Service ist zu Beginn gestoppt und disabled.\\

Starte den Service mit folgendem Befehl:
\begin{lstlisting}[style=BashInputStyle]
	 sudo systemctl start strongMan.service
\end{lstlisting}

Lasse den Service automatisch starten beim Systemstart:
\begin{lstlisting}[style=BashInputStyle]
	 sudo systemctl enable strongMan.service
\end{lstlisting}
Es werden Root Rechte benötigt, um Änderungen am Systemd zu machen.

\subsubsection{remove-service}
\begin{lstlisting}[style=BashInputStyle]
	 sudo ./setup.py remove-service
\end{lstlisting}
Entfernt den Systemd Service.

\subsubsection{migrate}
\begin{lstlisting}[style=BashInputStyle]
	 ./setup.py migrate [-dm]
\end{lstlisting}
Migriert die Django Datenbank, wenn Änderungen vorhanden sind. Siehe Django Migrations\footnote{\url{https://github.com/strongswan/strongswan/tree/master/src/libcharon/plugins/vici}}.
\begin{itemize}
	\item -dm | --delete-migrations
	\begin{itemize}
	    \item Löscht alle alten Migrationsskripte und die Sqlite Datenbank.
	\end{itemize}
\end{itemize}