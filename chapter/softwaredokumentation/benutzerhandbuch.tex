\section{Benutzerhandbuch}

\section{Installation}
\subsubsection{Anforderungen}
\begin{itemize}
	\item strongSwan mit Vici plugin (>= v5.4.0)
	\item python3 =<
	\item pip3 =<
	\item git
	\item virtualenv
\end{itemize}
\subsubsection{Installation}
\begin{lstlisting}[style=BashInputStyle]
	 git clone https://github.com/Sebubu/strongMan.git
	 cd strongMan
	 ./setup.py install
\end{lstlisting}

\subsubsection{Starten}
\begin{lstlisting}[style=BashInputStyle]
	 ./run.py
\end{lstlisting}

\subsubsection{Systemd Service hinzufügen (Optional)}
Füge einen Systemd Service hinzu, der strongMan bei jedem Systemstart automatisch startet.
\begin{lstlisting}[style=BashInputStyle]
    sudo ./setup.py add-service # Adds the service
    sudo systemctl enable strongMan.service
    sudo systemctl start strongMan.service
\end{lstlisting}

\subsection{Deinstallation}
\subsubsection{Systemd Service entfernen (Optional)}
Entferne den Systemd Service, wenn installiert.
\begin{lstlisting}[style=BashInputStyle]
    sudo ./setup.py remove-service
\end{lstlisting}

\subsubsection{Entferne Programmordner}
Der Programmordner kann einfach gelöscht werden.
\begin{lstlisting}[style=BashInputStyle]
    rm -rf strongMan/
\end{lstlisting}

